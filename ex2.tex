\chapter{Exercise 2: Creating A Multi-Table Database}

Creating one table isn't too useful.  I want you to now make 3 tables that 
you can store data into:

\begin{code}{ex2.sql}
\begin{Verbatim}
<< d['code/ex2.sql'] >>
\end{Verbatim}
\end{code}

In this file you are making tables for two types of data, and then
"linking" them together with a third table.  People call these 
"linking" tables "relations", but very pedantic people with no
lives call all tables "relations" and enjoy confusing people who
just want to get their jobs done.  In my book, tables that have data
are "tables", and tables that link tables together are called "relations".

There isn't anything new here, except when you look at \ident{person\_pet}
you'll see that I've made two columns: \ident{person\_id} and \ident{pet\_id}.
How you would link two tables together is simply \emph{insert} a row
into \ident{person\_pet} that had the values of the two row's id columns
you wanted to connect.

We'll get into actually inserting data like this in the next few exercises.

\section{What You Should See}

You run this SQL script in the same way as before, but you specify
\file{ex2.db} instead of \file{ex1.db}.  As usually there's no output,
but this time I want you to open the database and use the 
\ident{.schema} command to dump it:

\begin{code}{ex2.out}
\begin{Verbatim}
<< d['code/ex2.out'] >>
\end{Verbatim}
\end{code}

The "schema" should match what you typed in.

\section{Extra Credit}

\begin{enumerate}
\item In these tables I made a 3rd relation table to link them.  How would you
    get rid of this relation table \ident{person\_pet} and put that
    information right into \ident{person}?  What's the implication of this
    change?
\item If you can put one row into \ident{person} pet, can you put more than
    one?  How would you record a crazy cat lady with 50 cats?
\item Create another table for the cars people might own, and create its
    corresponding relation table.
\item Search for "sqlite3 datatypes" in your favorite search engine and 
    go read the "Datatypes In SQLite Version 3" document.  Take notes on
    what types you can use and other things that seem important.  We'll
    cover more later.
\end{enumerate}

\section{Portability Notes}

Databases have a lot of options for specifying the keys in these relations,
but for now we'll keep it simple.

