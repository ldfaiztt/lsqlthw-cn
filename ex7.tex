\chapter{Exercise 7: Deleting Data}

This is the simplest exercise, but I want you to think for a second before
typing the code in.  If you had "SELECT * FROM" for \ident{SELECT}, and
"INSERT INTO" for \ident{INSERT}, then how would you write the \ident{DELETE}
format?  You can probably glance down but try to guess at what it would
be then look.

\begin{code}{ex7.sql}
\begin{Verbatim}
<< d['code/ex7.sql'] >>
\end{Verbatim}
\end{code}

I'm simply implementing a very complex update of the robot by deleting 
him and then putting the record back but with \ident{dead=0}.  In later
exercises I'll show you how to use \ident{UPDATE} to do this, so don't
consider this to be the real way you'd do an update.

Most of the lines in this script are already familiar to you, with the
exception of line 5.  Here you have the \ident{DELETE} and it has
nearly the same format as other commands.  You give \verb|DELETE FROM table WHERE tests| and a way to think about it is being like a \ident{SELECT} that
removes rows.  Anything that works in a \ident{WHERE} clause will work here.

\section{What You Should See}

I'm going to reconstruct the entire database from scratch by replaying all of
the exercises to this point that you need.  This shows you how your work 
so far should continue to work as you go through the exercises.

\begin{code}{ex7.out}
\begin{Verbatim}
<< d['code/ex7.out'] >>
\end{Verbatim}
\end{code}

Notice at the end I'm adding the \program{sqlite3 -echo} so you can see
what statements run and what they produce.

\section{Extra Credit}

\begin{enumerate}
\item Go through the output from your run and make sure you know what table
    is produced for which SQL commands and how they produced that output.
\item Combine all of \file{ex2.sql} through \file{ex7.sql} into one file and
    redo the above script so you just run this one new file to recreate
    the database.
\item At the top of this new .sql file, add \verb|DROP TABLE| commands to
    drop the tables you're about to recreate. Now your script can run without
    you needing to \verb|rm ex3.db|.  You'll need to go look up the syntax
    for \ident{DROP TABLE}.
\item Add onto the script to delete other pets and insert them again with
    new values.  Remember that this is \emph{not} how you normally update
    records and is only for the exercise.
\end{enumerate}

