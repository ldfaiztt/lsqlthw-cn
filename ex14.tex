\chapter{Exercise 14: Basic Transactions}

Imagine if the SQL in your last exercise had an error half-way through it's run
and it aborted.  You may have even ran into this problem, and then you see that
your database is not seriously broken.  You've been getting away with this
because you have a big \file{code.sql} file that rebuilds your database, but in
a real situation you can't trash your whole database when you mess up.

What you need to make your script safer is the \ident{BEGIN}, \ident{COMMIT},
and \ident{ROLLBACK} commands.  These start a transaction, which creates a
"boundary" around a group of SQL statements so you can abort them if they have
an error.  You start the transaction with \ident{BEGIN}, do your SQL, and then
when everything's good end the transaction with \ident{COMMIT}.  If you have
an error then you just issue \ident{ROLLBACK} to abort what you did.

For this exercise I want you to do the following:

\begin{enumerate}
\item Take your \file{ex13.sql} and copy it to \file{ex14.sql} so you can modify it.
\item Once you have that, put a \ident{BEGIN} at the top and a \ident{ROLLBACK}.  At
    the bottom.
\item Now run it and you'll see that it's as if your script didn't do
    anything.
\item Next, change the \ident{ROLLBACK} to be \ident{COMMIT} and run it
    again, and you'll see it works like normal.
\item Get rid of the \ident{BEGIN} and \ident{COMMIT} from your \file{ex14.sql} so
    it's back the way it was.
\item Now create an error by removing one of the \ident{TABLE} keywords from
    one of the lines.  This is so you can make it have an error and recover.
\end{enumerate}

Once you have this broken \file{ex14.sql} you'll play with it in the \program{sqlite3}
console to do a recovery:

\begin{code}{ex14 Session Output}
<< d['code/ex14.sqlite3-console|pyg|l'] >>
\end{code}

This one's long so I'm going to break it down so you can track what's going on:

\begin{description}
\item[1] I start up \program{sqlite3} so I can get into the console with \file{ex14.db}.
\item[7] I then \ident{.read} the \file{code.sql} file to setup the database like 
    normal but doing it from within sqlite3.
\item[11] I enter \ident{BEGIN} so I can start a transaction boundary.
\item[12] I run \ident{.read ex14.sql} to run it, but remember it has an error
    so it aborts.  My error was on line 5 but yours could be anywhere else.
\item[14] I then run \ident{.schema} so you can see that some of the changes
    actually were made, and I'll want to undo them.
\item[16] Since the last command had an error I run \ident{ROLLBACK} here to 
    abort all the things I did since the \ident{BEGIN} on line 11.
\item[18] To show that the database is back the way it should be, I do another
    \ident{.schema} and you can see all that junk is now gone.
\end{description}
    

\section{Extra Credit}

\begin{enumerate}
\item Read the instructions on \href{http://www.sqlite.org/lang_transaction.html}{SQLite3 transactions} to get an idea of what's possible.
\item Use transactions to bound some example UPDATES and INSERTS to see how those
    work too.
\item Try using the alternative syntax of \ident{BEGIN TRANSACTION}, \ident{COMMIT TRANSACTION}, and \ident{ROLLBACK TRANSACTION}.
\end{enumerate}

\section{Portability Notes}

Some databases don't have the same rollback and commit semantics as other databases.
Some also don't understand the syntax with the word \ident{TRANSACTION}.  They should
universally abort everything you did, but again look at your manual to confirm
this is true.

