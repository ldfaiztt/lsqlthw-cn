\chapter{Exercise 13: Migrating And Evolving Data}

This exercise will have you apply some skills you've learned.  I'll have you
take your database and "evolve" the schema to a different form.  You'll need to
make sure you know the previous exercise well and have your \file{code.sql}
working as we'll be.  If you don't have either of these then go back and get
everything straightened out.

To make sure you are in the right state to attempt this exercise, when you run
your code.sql you should be able to run \ident{.schema} like this:

\begin{code}{ex13 Session Output}
<< d['code/ex13.sh-session|pyg|l'] >>
\end{code}

Make sure your tables look like my tables, and if not then go back and 
remove any commands that are doing \ident{ALTER TABLE} or anything from
the last exercise.

\section{The Assignment}

What you're tasked with doing is the following list of changes to 
the database:

\begin{enumerate}
\item Add a \ident{dead} column to \ident{person} that's like the one in \ident{pet}.
\item Add a \ident{phone\_number} column to \ident{person}.
\item Add a \ident{salary} column to \ident{person} that is \ident{float}.
\item Add a \ident{dob} column to both \ident{person} and \ident{pet} that is a \ident{DATETIME}.
\item Add a \ident{purchased\_on} column to \ident{person\_pet} of type \ident{DATETIME}.
\item Add a \ident{parent} column that's an \ident{INTEGER} and holds the \ident{id}
    for this pet's parent.
\item Update the existing database records with the new column data using \ident{UPDATE} statements.  Don't forget about the \ident{purchased\_on} column in \ident{person\_pet} relation table to indicate when that person bought the pet.
\item Add 4 more people and 5 more pets and assign their ownership and what pet's
    are parents.  On this last part remember that you get the id of the parent, then
    set it in the \ident{parent} column.
\item Write a query that can find all the names of pets and their owners bought after
    2004. Key to this is to map the \ident{person\_pet} based on the \ident{purchased\_on} column to the \ident{pet} and \ident{parent}.
\item Write a query that can find the pets that are children of a given pet.  Again
    look at the \ident{pet.parent} to do this.  It's actually easy so don't over
    think it.
\end{enumerate}

You should do this by writing a \file{ex13.sql} file with these new things in it.
You then test it by resetting the database using \file{code.sql} and then running
\file{ex13.sql} to alter the database and run the \ident{SELECT} queries that
confirm you made the right changes.

Take your time working on this, and when you're done your schema should look
like this:

\begin{code}{ex13 Session Output}
<< d['code/ex13-result.sh-session|pyg|l'] >>
\end{code}

