\chapter{Exercise 12: Destroying And Altering Tables}

You've already encountered \ident{DROP TABLE} as a way to get rid of
a table you've created.  I'm going to show you another way to use it
and also how to add or remove columns from a table with \ident{ALTER TABLE}.

\begin{code}{ex12.sql}
<< d['code/ex12.sql|pyg|l'] >>
\end{code}

I'm doing some fake changes to the tables to demonstrate the commands, but
this is everything you can do in SQLite3 with the \ident{ALTER TABLE} and
\ident{DROP TABLE} statements.  I'll walk through this so you understand
what's going on:

\begin{description}
\item[ex21.sql:2] Use the \ident{IF EXISTS} modifier and the table will be dropped
    only if it's already there.  This suppresses the error you get when
    running you .sql script on a fresh database that has no tables.
\item[ex21.sql:5] Just recreating the table again to work with it.
\item[ex21.sql:13] Using \ident{ALTER TABLE} to rename it to \ident{peoples}.
\item[ex21.sql:16] Add a new column \ident{hatred} that is an \ident{INTEGER} to
    the newly renamed table \ident{peoples}.
\item[ex21.sql:19] Rename \ident{peoples} back to \ident{person} because that's
    a dumb name for a table.
\item[ex21.sql:21] Dump the schema for \ident{person} so you can see it has the
    new \ident{hatred} column.
\item[ex21.sql:24] Drop the table to clean up after this exercise.
\end{description}

\section{What You Should See}

If you run this script it should look something like this:

\begin{code}{ex12 Session Output}
<< d['code/ex12.sh-session|pyg|l'] >>
\end{code}

I've added some extra spacing so you can read it easier, and remember to 
pass in the \program{-echo} argument so it prints out what it's run.

\section{Extra Credit}

\begin{enumerate}
\item Update your \file{code.sql} file you've been putting all the code in so that it
    uses the \ident{DROP TABLE IF EXISTS} syntax.
\item Use \ident{ALTER TABLE} to add a \ident{height} and \ident{weight} column
    to \ident{person} and put that in your \file{code.sql} file.
\item Run your new \file{code.sql} script to reset your database and you should
    have no errors.
\end{enumerate}

\section{Portability Notes}

Typically \ident{ALTER TABLE} is a mashup of just about everything a database
vendor couldn't put into their SQL syntax.  Some databases will let you do
more with tables than other databases, so read up on the documentation and
see what's possible.

