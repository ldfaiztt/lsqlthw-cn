\chapter{Exercise 4: Insert Referential Data}

In the last exercise you filled in some tables with people and pets.
The only thing that's missing is who owns what pets, and that data 
goes into the \ident{person\_pet} table like this:

\begin{code}{ex4.sql}
<< d['code/ex4.sql|pyg|l'] >>
\end{code}

Again I'm using the explicit format first, then the implicit format.
How this works is I'm using the \ident{id} values from the person
row I want (in this case, 0) and the \ident{id} from the pet rows
I want (again, 0 for the Unicorn and 1 for the Dead Robot).  I then
insert one row into \ident{person\_pet} relation table for each 
"connection" between a person and a pet.

\section{What You Should See}

I'll just piggyback on the last exercise and run this right on the
\file{ex3.db} database to set these values:

\begin{code}{ex4.out}
\begin{Verbatim}
<< d['code/ex4.out'] >>
\end{Verbatim}
\end{code}


\section{Extra Credit}

\begin{enumerate}
\item Add the relationships for you and your pets.
\item Using this table, could a pet be owned by more than one person?  Is that logically possible?  What about the family dog? Wouldn't everyone in the family technically own it?
\item Given the above, and given that you have an alternative design that puts the \ident{pet\_id} in the \ident{person} table, which design is better for this situation?
\end{enumerate}


