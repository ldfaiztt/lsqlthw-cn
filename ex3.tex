\chapter{Exercise 3: Inserting Data}

You have a couple tables to work with, so now I'll have you put some data into them
using the \ident{INSERT} command:

\begin{code}{ex3.sql}
<< d['code/ex3.sql|pyg|l'] >>
\end{code}

In this file I'm using two different forms of the \ident{INSERT} command.
The first form is the more explicit style, and most likely the one you should
use.  It specifies the columns that will be inserted, followed by \ident{VALUES},
then the data to include.  Both of these lists (column names and values)
go inside parenthesis and are separated by commas.

The second version on line 7 is an abbreviated version that doesn't specify
the columns and instead relies on the implicit order in the table.  This form
is dangerous since you don't know what column your statement is actually 
accessing, and some databases don't have reliable ordering for the columns.
It's best to only use this form when you're really lazy.

\section{What You Should See}

I'm going to reuse the \file{ex2.sql} file from the previous exercise to
recreate the database so you can put data into it.  This is what it looks
like when I run it:

\begin{code}{ex3.out}
\begin{Verbatim}
<< d['code/ex3.out'] >>
\end{Verbatim}
\end{code}

In the first line I just make \file{ex3.db} from the \file{ex2.sql} file.
Then I add the \ident{-echo} argument to \program{sqlite3} so that it
prints out what it is doing.  After that the data is in the database
and ready to query.

\section{Extra Credit}

\begin{enumerate}
\item Insert yourself and your pets (or imaginary pets like I have).
\item If you changed the database in the last exercise to not have the 
    \ident{person\_pet} table then make a new database with that schema,
    and insert the same information into it.
\item Go back to the list of data types and take notes on what format
    you need for the different types. For example, how many ways can you
    write TEXT data.
\end{enumerate}

\section{Portability Notes}

As I mentioned in the last exercise, database vendors tend to add lock-in
to their platforms by extending or altering the data types used.  They'll
subtly make \emph{their} TEXT columns a little different here, or \emph{their}
DATETIME columns are called TIMESTAMP and take a different format.  Watch
out for this when you use a different database.
